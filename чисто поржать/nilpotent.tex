\documentclass[14pt]{extarticle}
\usepackage{extsizes}
\usepackage{geometry}
\geometry{margin=0.5in}

%% for images
\usepackage{graphicx}
\graphicspath{ {images/} }

%% language support
\usepackage[T1,T2A]{fontenc}
\usepackage[utf8]{inputenc}
\usepackage[english,russian]{babel}

\usepackage{amsmath}
\usepackage{tikz}

%% hyperrefs
\usepackage{hyperref}
\hypersetup{
    colorlinks,
    citecolor=black,
    filecolor=black,
    linkcolor=black,
    urlcolor=black
}

\usepackage{xcolor}

\title{2024}
\author{Куся}
\date{00.00.2024}

\begin{document}
\maketitle
\tableofcontents

\section{Nilpotency considerations}
Let $\mathcal{J}_k$ be nilpotent matrix of order $k$. It looks like
\[\begin{pmatrix}
    0 & 1 & 0 & ... & 0   \\
    0 & 0 & 1 & ... & ... \\
    ... & 0 & ... & ... & 0 \\
    ... & ... & ... & 0 & 1 \\
    0 & ... & ... & 0 & 0   \\
\end{pmatrix}
\] \\
The main feature is that nilpotent matrices are degrees of each other. 
Let's prove the main thing: that
$\mathcal{J}_k$ is a nilpotent matrix of order $k$.
\\
Proof: \\
\textbf{Base case:} the \(\mathcal{J}_1 = \begin{pmatrix}
    0\\
\end{pmatrix}\) is nilpotent matrix of order 1.\\
\textbf{Induction step:}\\
Consider matrices of order $n$. 
Notice that $\mathcal{J}_m * \mathcal{J}_n = \mathcal{J}_{m-1}, \ m \leq n$
(because the last row of $\mathcal{J}_m$ always gives zero, and all 
other rows match the columns of larger matrix). 
\[\begin{pmatrix}
    0 & 0 & 0 & \textcolor{red}{1} & 0 & 0 \\
    0 & 0 & 0 & 0 & \textcolor{red}{1} & 0 \\
    0 & 0 & 0 & 0 & 0 & 1 \\
    0 & 0 & 0 & 0 & 0 & 0 \\
    0 & 0 & 0 & 0 & 0 & 0 \\
    0 & 0 & 0 & 0 & 0 & 0 \\
\end{pmatrix}
\begin{pmatrix}
    0 & 1 & 0 & 0 & 0 & 0 \\
    0 & 0 & 1 & 0 & 0 & 0 \\
    0 & 0 & 0 & 1 & 0 & 0 \\
    0 & 0 & 0 & 0 & \textcolor{red}{1} & 0 \\
    0 & 0 & 0 & 0 & 0 & \textcolor{red}{1} \\
    0 & 0 & 0 & 0 & 0 & 0 \\
\end{pmatrix} = 
\begin{pmatrix}
    0 & 0 & 0 & 0 & \textcolor{red}{1} & 0 \\
    0 & 0 & 0 & 0 & 0 & \textcolor{red}{1} \\
    0 & 0 & 0 & 0 & 0 & 0 \\
    0 & 0 & 0 & 0 & 0 & 0 \\
    0 & 0 & 0 & 0 & 0 & 0 \\
    0 & 0 & 0 & 0 & 0 & 0 \\
\end{pmatrix}
\] \\
$$ (\mathcal{J}_k * \mathcal{J}_k) * ... * \mathcal{J}_k 
= \mathcal{J}_{k-1}  * ... * \mathcal{J}_k = ...$$
$\mathcal{J}_k^k = \mathcal{J}_{k-1} * \mathcal{J}_k^{k-2} = 
\mathcal{J}_{k-(d-1)} * \mathcal{J}_k^{k-d} = 
\mathcal{J}_1 * \mathcal{J}_k^0 \ \ (d = k) = 0 * E = 0 \ \ \forall k > 0,
\ q.e.d.$

\end{document}
