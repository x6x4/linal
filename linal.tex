\documentclass[14pt]{extarticle}
\usepackage{extsizes}
\usepackage{geometry}
\geometry{margin=0.5in}

%% for images
\usepackage{graphicx}
\graphicspath{ {images/} }

%% language support
\usepackage[T1,T2A]{fontenc}
\usepackage[utf8]{inputenc}
\usepackage[english,russian]{babel}

\usepackage{amsmath}
\usepackage{tikz}

%% hyperrefs
\usepackage{hyperref}
\hypersetup{
    colorlinks,
    citecolor=black,
    filecolor=black,
    linkcolor=black,
    urlcolor=black
}

\title{Линал 2024}
\author{Ну я}
\date{19.01.2024}

\setcounter{secnumdepth}{0}
\begin{document}
\maketitle
\tableofcontents

\section{22. 
    Серии векторов относительно линейного оператора (ЛО). Жорданова клетка. 
    Вид матрицы ЛО в базисе из серий векторов. Понятие линейной независимости 
    системы векторов над подпространством и его взаимосвязь 
    с понятием линейной независимости. Примеры.
}

Последовательность векторов $\bar{a_1}, ..., \bar{a_m}$
пространства R называется \textbf{серией} с собственным значением 
$\lambda$ относительно преобразования $A$, если выполнены соотношения
$$ a_1 \neq 0; \ Aa_1 = \lambda a_1, \ Aa_2 = \lambda a_2 + a_1, \ ..., \
Aa_m = \lambda a_m + a_{m-1} $$
\\
\textbf{Теорема о приведении матрицы к жордановой форме:} \\
Существует базис пространства R, состоящий из всех векторов одной или 
нескольких серий относительно преобразования $A$. 
\\\\
В построенном согласно теореме базисе преобразованию $A$ соответствует 
некая новая матрица $B = (b^i_j)$, имеющая особо простую форму, 
называемую \textbf{жордановой}. \\
\textbf{Жорданова клетка} порядка $m$ с собственным значением $\lambda$ -
квадратная матрица $(b^i_j)$ порядка $m$, определяемая соотношениями \\
$ b^i_i = \lambda, \ i = 1, ... , m; \\ b^i_{i+1} = 1, \ i = 1, \ ... , \
m - 1; \\ b^i_j = 0, \ j - i < 0, \ j - i > 1 $, \\ т.е. матрица 

\[\begin{pmatrix}
    \lambda & 1 & 0 & ... & 0 & 0 \\
    0 & \lambda & 1 & ... & 0 & 0 \\
    0 & 0 & \lambda & ... & 0 & 0 \\
    . & . & . & . & . & .         \\
    0 & 0 & 0 & ... & \lambda & 1 \\
    0 & 0 & 0 & ... & 0 & \lambda \\
\end{pmatrix}\]
\\
Оказывается, что для каждой квадратной матрицы $A$ порядка $n$ можно
подобрать такую невырожденную квадратную матрицу $S$, что матрица
$B = SAS^{-1}$, получаемая из матрицы $A$ путём трансформации 
матрицей $S$, имеет \textbf{жорданову форму}. Т.е. \textsl{состоит из 
одной или нескольких жордановых клеток, расположенных по её главной 
диагонали, в то время как все элементы её, не входящие в жордановы 
клетки, равны нулю.} 

\section{23.	
    Теорема о существовании канонического базиса корневого подпространства 
    линейного оператора.
}

\section{29.	
    Понятие функции от матрицы. Теорема о существовании и единственности 
    интерполяционного многочлена Лагранжа-Сильвестра. 
    Пример вычисления функции от матрицы.
}

Пусть $\Delta(x) = (x-\lambda_1)^{k_1} (x-\lambda_2)^{k_2} ... (x-\lambda_r)^{k_r}, 
\ k_i > 0, \ i = 1, ... , r, \ k_1 + k_2 + ... + k_r = k$ 
- минимальный аннулирующий многочлен матрицы $A$, причем 
$\lambda_1, \lambda_2, ... , \lambda_r$ - его попарно различные корни. 
Они же составляют совокупность всех собственных значений матрицы $A$. \\
Говорят, что \textbf{на спектре матрицы} $A$ задана функция $f$, если 
каждому собственному значению $\lambda_i$ матрицы $A$ поставлена в соответствие
последовательность чисел $$ f^{(0)} (\lambda_i), \ f^{(1)} (\lambda_i), \ ..., 
\ f^{(k_i - 1)} (\lambda_i), \ i = 1, ... , r \ \ (*)$$
Если $f(z)$ - некоторая функция комплексного переменного $z$, то, понимая 
под числами (*) значение самой функции и её производных до порядка $k_i - 1$
в точке $\lambda_i$, мы получаем функцию, заданную на спектре матрицы $A$. 
Если для двух функций комплексного переменного $z$ значения (*) соответственно 
совпадают, то говорят, что эти две функции \textbf{совпадают на спектре 
матрицы} $A$. Оказывается, что два многочлена $g(z)$ и $h(z)$ тогда и только 
тогда совпадают на спектре матрицы $A$, когда $g(A) = h(A)$.\\\\
\textbf{Теорема о существовании и единственности интерполяционного многочлена 
Лагранжа-Сильвестра:} \\
Каковы бы ни были произвольно заданные числа (*), 
всегда существует единственный многочлен $\phi(z)$ степени $\leq k - 1$, 
значения которого на спектре матрицы $A$ совпадают с числами (*), т.е. 
$$\phi^{(j)}(\lambda_i) = f^{(j)}(\lambda_i), \ j = 0, ... , k_i - 1, \
i = 1, ... , r \ \ (**)$$ (т.н. \textbf{интерполяционный многочлен}). 
При этом коэффициенты многочлена $\phi(z)$ являются
линейными функциями величин (*) и потому непрерывно зависят от них. \\
\textbf{Док-во:} \\
Положим $d(z) = g(z) - h(z)$. Если $g(A) = h(A)$,
то $d(A) = 0$. Далее, если значения многочленов $g(z)$ и $h(z)$ на спектре 
матрицы $A$ совпадают, то функция $d(z)$ на спектре матрицы $A$ обращается в нуль.
Таким образом, чтобы доказать ту часть утверждения, которая относится к 
многочленам $g(z)$ и $h(z)$, достаточно доказать, что многочлен $d(z)$ тогда 
и только тогда аннулирует матрицу $A$, когда он обращается в нуль на 
спектре этой матрицы. Докажем это. \\
Допустим, что многочлен $d(z)$ аннулирует матрицу $A$. Тогда он делится на 
многочлен $\Delta(z)$ и потому имеет число $\lambda_i$ своим корнем кратности 
не меньше $k_i$, а из этого следует, что он обращается в нуль на спектре 
матрицы $A$. Если многочлен $d(z)$ обращается в нуль на спектре матрицы $A$, 
то он имеет число $\lambda_i$ своим корнем кратности не меньше $k_i$ и потому
делится на многочлен $\Delta(z)$, откуда следует, что $d(A) = 0$. \\
Докажем теперь ту часть, которая относится к функции $\phi(z)$. Совокупность 
соотношений (**) можно рассматривать как систему линейных уравнений относительно
коэффициентов многочлена $\phi(z)$ с $k$ уравнениями и $k$ неизвестными. Тогда
для доказательства достаточно установить, что детерминант этой системы отличен
от нуля, для чего достаточно доказать, что что в случае обращения в нуль 
правых частей этих уравнений решением будет лишь нулевой многочлен $\phi(z)$.
В случае обращения в нуль правых частей уравнений (**) многочлен $\phi(z)$
обращается в нуль на спектре матрицы $A$ и потому в силу доказанного ранее 
делится на многочлен $\Delta(z)$, а так как он имеет степень не выше $k-1$, 
то он тождественно равен нулю. Таким образом, теорема доказана.\\
\\
Пример вычисления функции от матрицы:\\
$A = \begin{pmatrix}
    4 & 4\\ 
    -1 & 0
  \end{pmatrix}, \ f(A) = e^A - ?$\\\\
Решение:\\\\
$\Delta(\lambda) =\begin{pmatrix}
    4 - \lambda & 4\\ 
    -1 & 0 - \lambda
  \end{pmatrix} = \lambda^2 - 4\lambda + 4 = (\lambda-2)^2$\\\\
  Степень многочлена 2, значит, интерполяционный многочлен линейный:
  $$ p(\lambda) = p_1\lambda + p_0 $$
  Приравниваем производные: \\\\
  $\begin{cases}
    f(2) = p(2)\\
    f^{'}(2) = p^{'}(2)
  \end{cases} \rightarrow 
  \begin{cases}
    e^2 = 2p_1 + p_0\\
    e^2 = p_1
  \end{cases}$\\\\\\
  Получаем $ p(\lambda) = e^2\lambda - e^2 $. Тогда \\\\
  $f(A) = e^A = p(A) = e^2 * \begin{pmatrix}
    4 & 4\\ 
    -1 & 0
  \end{pmatrix} - e^2 * \begin{pmatrix}
    1 & 0\\ 
    0 & 1
  \end{pmatrix} = e^2 * \begin{pmatrix}
    3 & 4\\ 
    -1 & -1
  \end{pmatrix}$

\section{30.	
    Теорема об общем виде интерполяционного многочлена Лагранжа-Сильвестра.
}

\end{document}
